\documentclass[11pt]{report}
\usepackage[utf8]{inputenc}
\usepackage[margin=1in]{geometry}
\usepackage{amsfonts,amsthm,amsmath}
\usepackage{mathtools}
\usepackage{complexity}
\usepackage[chapter]{algorithm}
\usepackage{algpseudocode,caption} 
\usepackage{graphicx}
\usepackage{relsize}
\usepackage{tikz}  %TikZ central library is called.
\usepackage{tkz-graph}
\usepackage{tkz-berge}
%\usepackage{tikz-network}
\usetikzlibrary{automata,positioning,calc}

\usepackage{tocloft}
\usepackage{palatino}

\RequirePackage[colorlinks=true]{hyperref}
\hypersetup{
  linkcolor=[rgb]{0.3,0.3,0.6},
  citecolor=[rgb]{0.2, 0.6, 0.2},
  urlcolor=[rgb]{0.6, 0.2, 0.2}
}

\usepackage{setspace}
\onehalfspacing


\usepackage{xcolor}
\usepackage{color}
\definecolor{delta}{rgb}{0,0.2,0}
\definecolor{gamma}{rgb}{0,0,0.2}
\definecolor{beta}{rgb}{0.2,0,0}
\definecolor{alpha}{rgb}{0.8,0,0}
\newcommand{\wt}[1]{\widetilde{#1}}
\newcommand{\sse}{\subseteq}
\newcommand{\zo}{\{0,1\}}
\newcommand{\zon}{\zo^n}
\newcommand{\aphantom}{\vphantom{2^2}}
\newcommand{\aaphantom}{\vphantom{2^{2^2}}}
\newcommand{\defn}{\stackrel{\text{\tiny def}}{=}}

% left-right wrappers
\newcommand{\set}[1]{\left\{ #1 \right\}}
\newcommand{\card}[1]{\left|#1 \right|}
\newcommand{\mytilde}[1]{\overset{\sim}{#1}}

% latin
\newcommand{\etal}{\textit{et al}.\@\xspace}
\newcommand{\ie}{i.e.}
\usepackage[all]{xy}
\usepackage{setspace}
\usepackage{amssymb}
\usepackage{amsmath}
\input{preamble.tex}
\usepackage{collect}
\usepackage{wrapfig}
\input{ps-macros.tex}
\renewcommand{\E}{{\mathbb E}}
\title{{\huge CS5130 : Mathematical Tools for Theoretical Computer Science} \\[3mm]
{\LARGE(Scribe Lecture Notes)}\\[1cm]}
\author{{\Large Lecturer : {\sc Jayalal Sarma}} \\[3mm]
Department of Computer Science and Engineering \\[1mm]
Indian Institute of Technology Madras (IITM)\\[1mm]
Chennai, India}
\date{Last updated on : \today}


\begin{document}
\maketitle
\setcounter{page}{1}

\newpage
\pagenumbering{roman}  % Roman numbering in intro portion.
\input{preface.tex}

\newpage
\listofscribe
\newpage
\tableofcontents
\newpage
\listoftodos
\pagenumbering{arabic}
\setcounter{page}{1}
%\setstretch{1.1}

\input{week01.tex}
\input{week02.tex}
\input{week03.tex}
\input{week04.tex}
\input{week05.tex} 
%\input{problemsets.tex}

\newpage
\chapter{Supplementary Material}

\section{Curiosity Collection}

Here we list down all the "out of curious" questions that we discussed (sometimes even not discussed) in the class (and hence in this document).
\includecollection{curious.tmp}

\newpage
\section{Exercises}
\setcounter{excount}{0}
\includecollection{ex.tmp}

\newpage

%% Convention : Call all problem set collection names with psXX.tmp This helps in managing the auxillary files created by collect package.Give back reference to problems in lectures.
%\def\psetbackref{0}

\section{Problem Sets}

\subsection{~Problem Set \#1}
\begin{enumerate}[(1)]
  \includecollection{ps1.tmp}
\end{enumerate}

%\newpage
%\section{~Problem Set \#2}
%
%\begin{enumerate}[(1)]
%\includecollection{ps2.tmp}
%\end{enumerate}
%
%\newpage
%\section{~Problem Set \#3}
%
%\begin{enumerate}[(1)]
%\includecollection{ps3.tmp}
%\end{enumerate}
%
%\newpage
%\section{~Problem Set \#4}
%
%\begin{enumerate}[(1)]
%\includecollection{ps4.tmp}
%\end{enumerate}

\bibliographystyle{apalike}
\bibliography{references}

\end{document}

